\documentclass[a4paper,12pt]{article}
\usepackage[utf8]{inputenc}
\usepackage[T1]{fontenc}
\usepackage[left=1.8cm, right=1.8cm, top=1.2cm, bottom=1.5cm]{geometry}
\usepackage{lmodern}
\usepackage{enumitem} % Keep enumitem for potential customization like leftmargin
\usepackage{verbatim}
\usepackage[italian]{babel}
\usepackage{listings}
\usepackage{makecell}
\usepackage{array} % Added for table column definitions
\usepackage{longtable} % Added for tables that might span pages
\usepackage{booktabs} % Added for better table rules
\usepackage{xcolor} % Added for colors
\usepackage{colortbl} % Added for colored table rows
\usepackage{hyperref} % Should generally be loaded last or close to last

% Define colors
\definecolor{lightgray}{gray}{0.9}

% Listings configuration
\lstset{
    basicstyle=\ttfamily,
    breaklines=true,
    language=bash % Default language for listings, can be overridden
}

% Improve array stretch for tables - Keep default or adjust slightly if needed
\renewcommand{\arraystretch}{1.4} % Adjusted slightly for booktabs

\title{Documentazione del Progetto: Visite Guidate}
\author{Felappi Giorgio e Barbetti Daniel}
\date{\today}

\begin{document}

    \maketitle
    \tableofcontents
    \newpage


    \section{Introduzione}

    \subsection{Contesto e Obiettivi del Progetto}
    L'applicazione ha lo scopo di far incontrare la domanda e l'offerta di visite guidate, mettendo in comunicazione le organizzazioni locali, che propongono visite su luoghi di interesse (naturale, storico, culturale), con i fruitori e i volontari.

    \bigskip % Added some space before the list
    \textbf{Tipologie di Utenti:}
% Removed [noitemsep] for standard spacing and guaranteed new lines
    \begin{itemize}
        \item \textbf{Configuratore}: Gestisce l'inserimento, aggiornamento e consultazione dei dati relativi alle visite e alle guide volontarie.
        \item \textbf{Volontario}: È la guida che, previa accreditamento, dichiara le proprie disponibilità e gestisce la propria agenda di visite.
        \item \textbf{Fruitore}: È l'utente che iscrive il proprio gruppo alle visite proposte.
    \end{itemize}

    \subsection{Ambito del Documento}
    Questo documento raccoglie:
% Removed [noitemsep]
    \begin{itemize}
        \item I casi d'uso (sia in forma testuale che tramite diagrammi UML).
        \item I diagrammi UML delle classi e, ove necessario, i diagrammi di sequenza.
        \item Il manuale di installazione e uso.
    \end{itemize}

    \newpage


    \section{Documentazione di Progetto per ogni Versione}

%--------------------------------------------------
% Versione 1
%--------------------------------------------------

    \subsection{Versione 1: Back-end Iniziale per il Configuratore}
    \textbf{Casi d'Uso:}

% --- UC V1-01: Login Configuratore ---
    \begin{longtable}{@{} p{0.2\textwidth} p{0.75\textwidth} @{}}
        \toprule
        \rowcolor{lightgray}
        \multicolumn{2}{c}{\textbf{Use Case: Login Configuratore}} \\
        \midrule
        \textbf{ID}        & V1-01                                                               \\
        \midrule
        \textbf{Versione}  & 1                                                                   \\
        \midrule
        \textbf{Attore(i)} & Configuratore (non autenticato)                                     \\
        \midrule
        \textbf{Obiettivo} & Il configuratore accede al sistema fornendo le proprie credenziali. \\
        \midrule
        \textbf{Precondizioni} &
% Removed compact spacing options, kept leftmargin for alignment
        \begin{itemize}[leftmargin=*]
            \item Il configuratore non è autenticato.
            \item Esistono credenziali predefinite o personali per il configuratore.
        \end{itemize} \\
        \midrule
        \textbf{\makecell[l]{Scenario \\Principale}} &
% Removed compact spacing options, kept leftmargin for alignment
        \begin{enumerate}[leftmargin=*]
            \item Il configuratore avvia la procedura di login.
            \item Il sistema richiede username e password.
            \item Il configuratore inserisce username e password.
            \item Il sistema verifica le credenziali.
            \item Le credenziali sono corrette e la password è già stata cambiata in passato (non è il primo accesso assoluto).
            \item Il sistema autentica il configuratore.
        \end{enumerate} \\
        \midrule
        \textbf{\makecell[l]{Scenario \\Alternativo}} &
        \textbf{5.a}: Le credenziali inserite non sono corrette. Il sistema visualizza un messaggio di errore e richiede nuovamente le credenziali (torna al passo 2). \\ \addlinespace % Added space for readability
        \textbf{5.b}: Le credenziali sono corrette, ma la password non è mai stata cambiata (primo accesso con credenziali predefinite/temporanee). Il sistema forza il cambio password (vedi UC V1-03). \\
        \midrule
        \textbf{Postcondizioni} &
% Removed compact spacing options, kept leftmargin for alignment
        \begin{itemize}[leftmargin=*]
            \item Se la fase di creazione corpo dati non è completata, il configuratore accede a tale fase (vedi UC V1-05).
            \item Se la fase di creazione corpo dati è completata, il configuratore accede al menu a regime (vedi UC V1-04).
            \item In caso di fallimento, lo stato del sistema rimane invariato (configuratore non autenticato).
        \end{itemize} \\
        \bottomrule
        \caption{UC V1-01: Login Configuratore} \label{uc:v1-01}
    \end{longtable}

% --- UC V1-02: Cambio Password (Regime) ---
    \newpage % Kept newpage as requested in previous iteration
    \begin{longtable}{@{} p{0.2\textwidth} p{0.75\textwidth} @{}}
        \toprule
        \rowcolor{lightgray}
        \multicolumn{2}{c}{\textbf{Use Case: Cambio Password (Regime)}} \\
        \midrule
        \textbf{ID}        & V1-02                                                   \\
        \midrule
        \textbf{Versione}  & 1                                                       \\
        \midrule
        \textbf{Attore(i)} & Configuratore (autenticato)                             \\
        \midrule
        \textbf{Obiettivo} & Il configuratore cambia la propria password di accesso. \\
        \midrule
        \textbf{Precondizioni} &
        \begin{itemize}[leftmargin=*]
            \item Il configuratore è autenticato.
            \item Il configuratore si trova nel menu a regime (non durante il primo accesso o la creazione corpo dati).
        \end{itemize} \\
        \midrule
        \textbf{\makecell[l]{Scenario \\Principale}} &
        \begin{enumerate}[leftmargin=*]
            \item Il configuratore seleziona l'opzione per cambiare la password.
            \item Il sistema richiede la nuova password (eventualmente con conferma).
            \item Il configuratore inserisce la nuova password.
            \item Il sistema valida e aggiorna la password del configuratore.
            \item Il sistema conferma l'avvenuto cambio.
        \end{enumerate} \\
        \midrule
        \textbf{\makecell[l]{Scenario \\Alternativo}} & \textbf{4.a}: La nuova password non rispetta i criteri di sicurezza (se definiti). Il sistema visualizza un errore e richiede nuovamente la password (torna al passo 2). \\
        \midrule
        \textbf{Postcondizioni} &
        \begin{itemize}[leftmargin=*]
            \item La password del configuratore è aggiornata.
            \item Il configuratore torna al menu a regime.
        \end{itemize} \\
        \bottomrule
        \caption{UC V1-02: Cambio Password (Regime)} \label{uc:v1-02}
    \end{longtable}

% --- UC V1-03: Cambio Password Obbligatorio (Primo Accesso) ---
    \newpage
    \begin{longtable}{@{} p{0.2\textwidth} p{0.75\textwidth} @{}}
        \toprule
        \rowcolor{lightgray}
        \multicolumn{2}{c}{\textbf{Use Case: Cambio Password Obbligatorio (Primo Accesso)}} \\
        \midrule
        \textbf{ID}        & V1-03                                                                                                                                                                    \\
        \midrule
        \textbf{Versione}  & 1                                                                                                                                                                        \\
        \midrule
        \textbf{Attore(i)} & Configuratore (autenticato)                                                                                                                                              \\
        \midrule
        \textbf{Obiettivo} & Il configuratore è obbligato a cambiare la password predefinita/temporanea al primo accesso.                                                                             \\
        \midrule
        \textbf{Precondizioni} &
        \begin{itemize}[leftmargin=*]
            \item Il configuratore si è appena autenticato per la prima volta (UC V1-01, Alt 5.b).
            \item Le funzionalità disponibili sono limitate al cambio password e logout.
        \end{itemize} \\
        \midrule
        \textbf{\makecell[l]{Scenario \\Principale}} &
        \begin{enumerate}[leftmargin=*]
            \item Il sistema informa l'utente della necessità di cambiare password.
            \item Il sistema richiede la nuova password (eventualmente con conferma).
            \item Il configuratore inserisce la nuova password.
            \item Il sistema valida e aggiorna la password del configuratore.
            \item Il sistema conferma l'avvenuto cambio e sblocca le funzionalità normali.
        \end{enumerate} \\
        \midrule
        \textbf{\makecell[l]{Scenario \\Alternativo}}                   & \textbf{3.a}: Il configuratore sceglie di effettuare il logout prima di cambiare la password. Al login successivo, si ritroverà in questa stessa condizione. \\ \addlinespace
        & \textbf{4.a}: La nuova password non rispetta i criteri di sicurezza (se definiti). Il sistema visualizza un errore e richiede nuovamente la password (torna al passo 2). \\
        \midrule
        \textbf{Postcondizioni} &
        \begin{itemize}[leftmargin=*]
            \item La password del configuratore è aggiornata.
            \item Il configuratore accede alla fase di creazione corpo dati (se non completata) o al menu a regime.
        \end{itemize} \\
        \bottomrule
        \caption{UC V1-03: Cambio Password Obbligatorio} \label{uc:v1-03}
    \end{longtable}

% --- UC V1-04: Menu a Regime ---
    \newpage
    \begin{longtable}{@{} p{0.2\textwidth} p{0.75\textwidth} @{}}
        \toprule
        \rowcolor{lightgray}
        \multicolumn{2}{c}{\textbf{Use Case: Menu a Regime}} \\
        \midrule
        \textbf{ID}             & V1-04                                                                                                   \\
        \midrule
        \textbf{Versione}       & 1                                                                                                       \\
        \midrule
        \textbf{Attore(i)}      & Configuratore (autenticato)                                                                             \\
        \midrule
        \textbf{Obiettivo}      & Il configuratore interagisce con il menu principale dell'applicazione dopo la fase di inizializzazione. \\
        \midrule
        \textbf{Precondizioni} &
        \begin{itemize}[leftmargin=*]
            \item Il configuratore è autenticato.
            \item La fase di creazione del corpo dati (UC V1-05) è stata completata.
        \end{itemize} \\
        \midrule
        \textbf{\makecell[l]{Scenario \\Principale}} &
        \begin{enumerate}[leftmargin=*]
            \item Il sistema presenta al configuratore le opzioni disponibili a regime per la Versione 1:
            % Use standard itemize for clarity
            \begin{itemize}
                \item Indicare date precluse (UC V1-11)
                \item Modificare numero massimo persone per iscrizione (UC V1-07)
                \item Visualizzare elenco volontari e tipi visita associati (UC V1-12)
                \item Visualizzare elenco luoghi (UC V1-13)
                \item Visualizzare elenco tipi visita per luogo (UC V1-14)
                \item Visualizzare elenco visite per stato (UC V1-15)
                \item Cambiare la propria password (UC V1-02)
                \item Effettuare Logout
            \end{itemize}
            \item Il configuratore seleziona un'opzione.
            \item Il sistema esegue l'azione corrispondente (invocando l'UC relativo).
            \item Al termine dell'azione, il sistema torna a presentare il menu (passo 1).
        \end{enumerate} \\
        \midrule
        \textbf{\makecell[l]{Scenario \\Alternativo}} & \textbf{2.a}: Il configuratore seleziona l'opzione di Logout. La sessione termina.                                               \\
        \midrule
        \textbf{Postcondizioni} & Il configuratore rimane nel menu a regime fino al logout.                                               \\
        \bottomrule
        \caption{UC V1-04: Menu a Regime} \label{uc:v1-04}
    \end{longtable}

% --- UC V1-05: Creazione Corpo Dati (Inizializzazione) ---
    \newpage
    \begin{longtable}{@{} p{0.2\textwidth} p{0.75\textwidth} @{}}
        \toprule
        \rowcolor{lightgray}
        \multicolumn{2}{c}{\textbf{Use Case: Creazione Corpo Dati (Inizializzazione)}} \\
        \midrule
        \textbf{ID}        & V1-05                                                                                         \\
        \midrule
        \textbf{Versione}  & 1                                                                                             \\
        \midrule
        \textbf{Attore(i)} & Configuratore (autenticato, solitamente il primo)                                             \\
        \midrule
        \textbf{Obiettivo} & Il configuratore imposta i dati iniziali fondamentali per il funzionamento dell'applicazione. \\
        \midrule
        \textbf{Precondizioni} &
        \begin{itemize}[leftmargin=*]
            \item Il configuratore è autenticato (tipicamente dopo UC V1-01 o V1-03).
            \item La fase di creazione corpo dati non è ancora stata completata (flag di sistema).
        \end{itemize} \\
        \midrule
        \textbf{\makecell[l]{Scenario \\Principale}} &
        \begin{enumerate}[leftmargin=*]
            \item Il sistema presenta le opzioni per l'inizializzazione.
            \item Il configuratore esegue \textbf{obbligatoriamente per primo} l'assegnazione dell'ambito territoriale (UC V1-06).
            \item Il configuratore esegue (in qualsiasi ordine, almeno una volta ciascuna):
            \begin{itemize} % Standard spacing
                \item Aggiunta Luoghi (UC V1-08)
                \item Aggiunta Tipi Visita (UC V1-09)
                \item Aggiunta Volontari (UC V1-10)
                \item Associazione TipoVisita-Luogo (implicito in UC V1-09/V1-10)
                \item Associazione Volontario-TipoVisita (implicito in UC V1-10)
                \item Assegnazione numero massimo persone per iscrizione (UC V1-07)
            \end{itemize}
            \item Il configuratore, una volta inseriti tutti i dati necessari, seleziona l'opzione per completare la fase di inizializzazione.
            \item Il sistema verifica che tutti i passi obbligatori siano stati eseguiti almeno una volta e che esista una configurazione minima valida (es. almeno un luogo con un tipo visita e un volontario associato).
            \item Il sistema marca la fase di creazione corpo dati come completata.
        \end{enumerate} \\
        \midrule
        \textbf{\makecell[l]{Scenario \\Alternativo}} & \textbf{5.a}: La verifica al passo 5 fallisce (mancano dati obbligatori o configurazione non valida). Il sistema informa il configuratore e rimane nella fase di creazione corpo dati (torna al passo 1). \\
        \midrule
        \textbf{Postcondizioni} &
        \begin{itemize}[leftmargin=*]
            \item Il corpo dati iniziale dell'applicazione è creato e salvato persistentemente.
            \item Il flag di sistema per la creazione corpo dati è impostato a "completato".
            \item Il configuratore viene portato al menu a regime (UC V1-04).
        \end{itemize} \\
        \bottomrule
        \caption{UC V1-05: Creazione Corpo Dati} \label{uc:v1-05}
    \end{longtable}

% --- UC V1-06: Assegnazione Ambito Territoriale ---
    \newpage
    \begin{longtable}{@{} p{0.2\textwidth} p{0.75\textwidth} @{}}
        \toprule
        \rowcolor{lightgray}
        \multicolumn{2}{c}{\textbf{Use Case: Assegnazione Ambito Territoriale}} \\
        \midrule
        \textbf{ID}        & V1-06                                                                                          \\
        \midrule
        \textbf{Versione}  & 1                                                                                              \\
        \midrule
        \textbf{Attore(i)} & Configuratore (autenticato)                                                                    \\
        \midrule
        \textbf{Obiettivo} & Il configuratore definisce l'ambito territoriale di competenza dell'applicazione (una tantum). \\
        \midrule
        \textbf{Precondizioni} &
        \begin{itemize}[leftmargin=*]
            \item Il configuratore è nella fase di creazione del corpo dati (UC V1-05).
            \item L'ambito territoriale non è ancora stato definito.
        \end{itemize} \\
        \midrule
        \textbf{\makecell[l]{Scenario \\Principale}} &
        \begin{enumerate}[leftmargin=*]
            \item Il configuratore seleziona l'opzione per assegnare l'ambito territoriale.
            \item Il sistema richiede l'inserimento dell'ambito (es. nome comune/area).
            \item Il configuratore inserisce l'informazione.
            \item Il sistema valida (es. non vuoto) e salva l'ambito territoriale in modo persistente e non modificabile.
            \item Il sistema conferma l'avvenuta assegnazione.
        \end{enumerate} \\
        \midrule
        \textbf{\makecell[l]{Scenario \\Alternativo}} & \textbf{4.a}: L'informazione inserita non è valida (es. stringa vuota). Il sistema visualizza un errore e richiede nuovamente l'inserimento (torna al passo 2). \\
        \midrule
        \textbf{Postcondizioni} &
        \begin{itemize}[leftmargin=*]
            \item L'ambito territoriale è salvato permanentemente.
            \item Il configuratore torna al menu della fase di creazione corpo dati.
        \end{itemize} \\
        \bottomrule
        \caption{UC V1-06: Assegnazione Ambito Territoriale} \label{uc:v1-06}
    \end{longtable}

% --- UC V1-07: Assegnazione/Modifica Numero Massimo Persone per Iscrizione ---
    \newpage
    \begin{longtable}{@{} p{0.2\textwidth} p{0.75\textwidth} @{}}
        \toprule
        \rowcolor{lightgray}
        \multicolumn{2}{c}{\textbf{Use Case: Gestione Max Persone Iscrizione}} \\
        \midrule
        \textbf{ID}        & V1-07                                                                                                                        \\
        \midrule
        \textbf{Versione}  & 1                                                                                                                            \\
        \midrule
        \textbf{Attore(i)} & Configuratore (autenticato)                                                                                                  \\
        \midrule
        \textbf{Obiettivo} & Il configuratore imposta o modifica il numero massimo di persone che un fruitore può iscrivere con una singola prenotazione. \\
        \midrule
        \textbf{Precondizioni} &
        \begin{itemize}[leftmargin=*]
            \item Il configuratore è autenticato.
            \item Il configuratore è nella fase di creazione corpo dati (UC V1-05) o nel menu a regime (UC V1-04).
        \end{itemize} \\
        \midrule
        \textbf{\makecell[l]{Scenario \\Principale}} &
        \begin{enumerate}[leftmargin=*]
            \item Il configuratore seleziona l'opzione per impostare/modificare il numero massimo di persone per iscrizione.
            \item Il sistema richiede l'inserimento del numero.
            \item Il configuratore inserisce un numero intero.
            \item Il sistema valida che il numero sia strettamente positivo (>= 1).
            \item Il sistema salva il valore in modo persistente.
            \item Il sistema conferma l'operazione.
        \end{enumerate} \\
        \midrule
        \textbf{\makecell[l]{Scenario \\Alternativo}} & \textbf{4.a}: Il valore inserito non è un numero intero o non è >= 1. Il sistema visualizza un errore e richiede nuovamente l'inserimento (torna al passo 2). \\
        \midrule
        \textbf{Postcondizioni} &
        \begin{itemize}[leftmargin=*]
            \item Il numero massimo di persone per iscrizione è salvato/aggiornato.
            \item Il configuratore torna al menu precedente (creazione corpo dati o regime).
        \end{itemize} \\
        \bottomrule
        \caption{UC V1-07: Gestione Max Persone Iscrizione} \label{uc:v1-07}
    \end{longtable}

% --- UC V1-08: Aggiunta Luogo ---
    \newpage
    \begin{longtable}{@{} p{0.2\textwidth} p{0.75\textwidth} @{}}
        \toprule
        \rowcolor{lightgray}
        \multicolumn{2}{c}{\textbf{Use Case: Aggiunta Luogo}} \\
        \midrule
        \textbf{ID}        & V1-08                                                           \\
        \midrule
        \textbf{Versione}  & 1 (Utilizzato in fase iniziale e poi esteso in V3 per regime)   \\
        \midrule
        \textbf{Attore(i)} & Configuratore (autenticato)                                     \\
        \midrule
        \textbf{Obiettivo} & Il configuratore aggiunge un nuovo luogo visitabile al sistema. \\
        \midrule
        \textbf{Precondizioni} &
        \begin{itemize}[leftmargin=*]
            \item Il configuratore è autenticato.
            \item Il configuratore è nella fase di creazione corpo dati (UC V1-05) o nel menu a regime (a partire da V3).
            \item L'ambito territoriale è stato definito (UC V1-06).
        \end{itemize} \\
        \midrule
        \textbf{\makecell[l]{Scenario \\Principale}} &
        \begin{enumerate}[leftmargin=*]
            \item Il configuratore seleziona l'opzione per aggiungere un luogo.
            \item Il sistema richiede i dati del luogo:
            \begin{itemize} % Standard itemize
                \item Nome/Identificativo univoco
                \item Descrizione (opzionale)
                \item Collocazione geografica (indirizzo o coordinate)
            \end{itemize}
            \item Il configuratore inserisce i dati richiesti.
            \item Il sistema valida i dati (es. nome univoco, non vuoto; formato collocazione corretto; appartenenza all'ambito territoriale - se verificabile).
            \item Il sistema salva il nuovo luogo in modo persistente.
            \item Il sistema conferma l'aggiunta.
        \end{enumerate} \\
        \midrule
        \textbf{\makecell[l]{Scenario \\Alternativo}} & \textbf{4.a}: I dati inseriti non sono validi (es. nome duplicato, campi obbligatori vuoti, formato errato). Il sistema visualizza un errore specifico e richiede nuovamente l'inserimento (torna al passo 2). \\
        \midrule
        \textbf{Postcondizioni} &
        \begin{itemize}[leftmargin=*]
            \item Il nuovo luogo è aggiunto al sistema.
            \item Il configuratore torna al menu precedente.
        \end{itemize} \\
        \bottomrule
        \caption{UC V1-08: Aggiunta Luogo} \label{uc:v1-08}
    \end{longtable}

% --- UC V1-09: Aggiunta Tipo Visita ---
    \newpage
    \begin{longtable}{@{} p{0.2\textwidth} p{0.75\textwidth} @{}}
        \toprule
        \rowcolor{lightgray}
        \multicolumn{2}{c}{\textbf{Use Case: Aggiunta Tipo Visita}} \\
        \midrule
        \textbf{ID}        & V1-09                                                                             \\
        \midrule
        \textbf{Versione}  & 1 (Utilizzato in fase iniziale e poi esteso in V3 per regime)                     \\
        \midrule
        \textbf{Attore(i)} & Configuratore (autenticato)                                                       \\
        \midrule
        \textbf{Obiettivo} & Il configuratore aggiunge un nuovo tipo di visita associato a un luogo esistente. \\
        \midrule
        \textbf{Precondizioni} &
        \begin{itemize}[leftmargin=*]
            \item Il configuratore è autenticato.
            \item Esiste almeno un luogo nel sistema (UC V1-08).
            \item Il configuratore è nella fase di creazione corpo dati (UC V1-05) o nel menu a regime (a partire da V3).
        \end{itemize} \\
        \midrule
        \textbf{\makecell[l]{Scenario \\Principale}} &
        \begin{enumerate}[leftmargin=*]
            \item Il configuratore seleziona l'opzione per aggiungere un tipo di visita.
            \item Il sistema richiede di selezionare/indicare il luogo a cui associare il tipo visita.
            \item Il configuratore seleziona il luogo.
            \item Il sistema richiede i dati del tipo visita:
            \begin{itemize} % Standard itemize
                \item Titolo univoco (nell'ambito del luogo)
                \item Descrizione
                \item Punto di incontro
                \item Periodo di programmabilità (data inizio, data fine)
                \item Giornate settimanali di programmabilità
                \item Ora di inizio
                \item Durata (minuti)
                \item Necessità biglietto (sì/no)
                \item Numero minimo partecipanti
                \item Numero massimo partecipanti
            \end{itemize}
            \item Il configuratore inserisce i dati richiesti.
            \item Il sistema valida i dati (es. titolo univoco per il luogo, date valide, numeri positivi, ora valida, non sovrapposizione oraria con altri tipi visita dello stesso luogo nello stesso giorno - come da pag. 5-6 PDF).
            \item Il sistema salva il nuovo tipo di visita associato al luogo.
            \item Il sistema chiede di associare almeno un volontario (gestito da UC V1-10/V3-04).
            \item Il sistema conferma l'aggiunta del tipo visita.
        \end{enumerate} \\
        \midrule
        \textbf{\makecell[l]{Scenario \\Alternativo}} &
        \textbf{3.a}: Il luogo selezionato non esiste. Errore. \\ \addlinespace
        \textbf{6.a}: I dati inseriti non sono validi (univocità, formato, regole di business come non sovrapposizione). Il sistema visualizza un errore specifico e richiede nuovamente l'inserimento (torna al passo 4). \\
        \midrule
        \textbf{Postcondizioni} &
        \begin{itemize}[leftmargin=*]
            \item Il nuovo tipo di visita è aggiunto al sistema e associato al luogo specificato.
            \item Il configuratore torna al menu precedente.
        \end{itemize} \\
        \bottomrule
        \caption{UC V1-09: Aggiunta Tipo Visita} \label{uc:v1-09}
    \end{longtable}

% --- UC V1-10: Aggiunta Volontario e Associazione a Tipo Visita ---
    \newpage
    \begin{longtable}{@{} p{0.2\textwidth} p{0.75\textwidth} @{}}
        \toprule
        \rowcolor{lightgray}
        \multicolumn{2}{c}{\textbf{Use Case: Gestione Volontari e Associazioni}} \\
        \midrule
        \textbf{ID}        & V1-10                                                                                                  \\
        \midrule
        \textbf{Versione}  & 1 (Utilizzato in fase iniziale e poi esteso in V3 per regime)                                          \\
        \midrule
        \textbf{Attore(i)} & Configuratore (autenticato)                                                                            \\
        \midrule
        \textbf{Obiettivo} & Il configuratore aggiunge un nuovo volontario al sistema e/o lo associa a un tipo di visita esistente. \\
        \midrule
        \textbf{Precondizioni} &
        \begin{itemize}[leftmargin=*]
            \item Il configuratore è autenticato.
            \item Esiste almeno un tipo di visita nel sistema (UC V1-09).
            \item Il configuratore è nella fase di creazione corpo dati (UC V1-05) o nel menu a regime (a partire da V3).
        \end{itemize} \\
        \midrule
        \textbf{\makecell[l]{Scenario \\Principale}} &
        \begin{enumerate}[leftmargin=*]
            \item Il configuratore seleziona l'opzione per aggiungere/associare un volontario.
            \item Il sistema chiede se si vuole aggiungere un nuovo volontario o associarne uno esistente.
            \item \textbf{Caso Nuovo Volontario:}
            \begin{enumerate}[label=\alph*., leftmargin=*] % Standard enumerate
                \item Il sistema richiede il nickname (univoco) e una password temporanea.
                \item Il configuratore inserisce i dati.
                \item Il sistema valida (nickname univoco, non vuoto).
                \item Il sistema salva il nuovo volontario.
            \end{enumerate}
            \item \textbf{Caso Volontario Esistente:}
            \begin{enumerate}[label=\alph*., leftmargin=*] % Standard enumerate
                \item Il sistema chiede di selezionare/indicare il volontario.
                \item Il configuratore seleziona il volontario.
            \end{enumerate}
            \item Il sistema chiede di selezionare/indicare il tipo di visita a cui associare il volontario.
            \item Il configuratore seleziona il tipo di visita.
            \item Il sistema valida che volontario e tipo visita esistano e non siano già associati.
            \item Il sistema salva l'associazione tra il volontario (nuovo o esistente) e il tipo di visita.
            \item Il sistema conferma l'operazione.
        \end{enumerate} \\
        \midrule
        \textbf{\makecell[l]{Scenario \\Alternativo}}                   & \textbf{3.c.a}: Nickname duplicato o non valido. Errore, torna a 3.a.                                          \\ \addlinespace
        & \textbf{4.a.a}: Volontario selezionato non esistente. Errore.                                           \\ \addlinespace
        & \textbf{6.a}: Tipo visita selezionato non esistente. Errore.                                       \\ \addlinespace
        & \textbf{7.a}: Associazione già esistente. Messaggio informativo. \\
        \midrule
        \textbf{Postcondizioni} &
        \begin{itemize}[leftmargin=*]
            \item Se aggiunto, il nuovo volontario esiste nel sistema.
            \item L'associazione tra il volontario specificato e il tipo visita specificato è creata/salvata.
            \item Ogni volontario deve essere associato ad almeno un tipo visita (verificato al termine della fase iniziale o gestito da V3).
            \item Ogni tipo visita deve avere almeno un volontario associato (verificato al termine della fase iniziale o gestito da V3).
            \item Il configuratore torna al menu precedente.
        \end{itemize} \\
        \bottomrule
        \caption{UC V1-10: Gestione Volontari e Associazioni} \label{uc:v1-10}
    \end{longtable}

% --- UC V1-11: Indicazione Date Precluse ---
    \newpage
    \begin{longtable}{@{} p{0.2\textwidth} p{0.75\textwidth} @{}}
        \toprule
        \rowcolor{lightgray}
        \multicolumn{2}{c}{\textbf{Use Case: Indicazione Date Precluse}} \\
        \midrule
        \textbf{ID}        & V1-11                                                                                \\
        \midrule
        \textbf{Versione}  & 1                                                                                    \\
        \midrule
        \textbf{Attore(i)} & Configuratore (autenticato)                                                          \\
        \midrule
        \textbf{Obiettivo} & Il configuratore indica le date future in cui nessuna visita può essere programmata. \\
        \midrule
        \textbf{Precondizioni} &
        \begin{itemize}[leftmargin=*]
            \item Il configuratore è autenticato e nel menu a regime (UC V1-04).
            \item La data corrente è compresa tra il giorno 16 del mese 'i' e il giorno 15 del mese 'i+1'.
        \end{itemize} \\
        \midrule
        \textbf{\makecell[l]{Scenario \\Principale}} &
        \begin{enumerate}[leftmargin=*]
            \item Il configuratore seleziona l'opzione per indicare le date precluse.
            \item Il sistema informa che le date da inserire si riferiscono al mese 'i+3'.
            \item Il sistema richiede l'inserimento di una data (o più date).
            \item Il configuratore inserisce una data nel formato richiesto.
            \item Il sistema valida che la data sia nel mese 'i+3' e sia formalmente corretta.
            \item Il sistema salva la data come preclusa per il mese 'i+3'.
            \item Il sistema chiede se si vogliono inserire altre date per il mese 'i+3'. Se sì, torna al passo 3.
            \item Il sistema conferma le operazioni.
        \end{enumerate} \\
        \midrule
        \textbf{\makecell[l]{Scenario \\Alternativo}}                   & \textbf{5.a}: La data inserita non è nel mese 'i+3' o non è valida. Il sistema visualizza un errore e richiede nuovamente l'inserimento (torna al passo 3).     \\ \addlinespace
        & \textbf{6.a}: La data è già stata inserita come preclusa. Messaggio informativo. \\
        \midrule
        \textbf{Postcondizioni} &
        \begin{itemize}[leftmargin=*]
            \item Le date indicate sono salvate come precluse per il mese di riferimento ('i+3').
            \item Queste date verranno considerate durante la pianificazione delle visite (V3).
            \item Il configuratore torna al menu a regime.
        \end{itemize} \\
        \bottomrule
        \caption{UC V1-11: Indicazione Date Precluse} \label{uc:v1-11}
    \end{longtable}

% --- UC V1-12: Visualizzazione Elenco Volontari e Tipi Visita Associati ---
    \newpage
    \begin{longtable}{@{} p{0.2\textwidth} p{0.75\textwidth} @{}}
        \toprule
        \rowcolor{lightgray}
        \multicolumn{2}{c}{\textbf{Use Case: Elenco Volontari e Associazioni}} \\
        \midrule
        \textbf{ID}        & V1-12                                                                                                        \\
        \midrule
        \textbf{Versione}  & 1                                                                                                            \\
        \midrule
        \textbf{Attore(i)} & Configuratore (autenticato)                                                                                  \\
        \midrule
        \textbf{Obiettivo} & Il configuratore visualizza l'elenco dei volontari registrati e i tipi di visita a cui ciascuno è associato. \\
        \midrule
        \textbf{Precondizioni} &
        \begin{itemize}[leftmargin=*]
            \item Il configuratore è autenticato e nel menu a regime (UC V1-04).
        \end{itemize} \\
        \midrule
        \textbf{\makecell[l]{Scenario \\Principale}} &
        \begin{enumerate}[leftmargin=*]
            \item Il configuratore seleziona l'opzione per visualizzare l'elenco dei volontari.
            \item Il sistema recupera l'elenco dei volontari e le relative associazioni ai tipi di visita.
            \item Il sistema visualizza l'elenco, mostrando per ogni volontario (nickname) la lista dei titoli dei tipi di visita associati.
        \end{enumerate} \\
        \midrule
        \textbf{\makecell[l]{Scenario \\Alternativo}} & \textbf{3.a}: Non ci sono volontari registrati. Il sistema visualizza un messaggio appropriato (es. "Nessun volontario registrato"). \\
        \midrule
        \textbf{Postcondizioni} &
        \begin{itemize}[leftmargin=*]
            \item Il configuratore ha visualizzato le informazioni richieste.
            \item Il configuratore torna al menu a regime.
        \end{itemize} \\
        \bottomrule
        \caption{UC V1-12: Elenco Volontari e Associazioni} \label{uc:v1-12}
    \end{longtable}

% --- UC V1-13: Visualizzazione Elenco Luoghi ---
    \newpage
    \begin{longtable}{@{} p{0.2\textwidth} p{0.75\textwidth} @{}}
        \toprule
        \rowcolor{lightgray}
        \multicolumn{2}{c}{\textbf{Use Case: Elenco Luoghi}} \\
        \midrule
        \textbf{ID}        & V1-13                                                                            \\
        \midrule
        \textbf{Versione}  & 1                                                                                \\
        \midrule
        \textbf{Attore(i)} & Configuratore (autenticato)                                                      \\
        \midrule
        \textbf{Obiettivo} & Il configuratore visualizza l'elenco dei luoghi visitabili inseriti nel sistema. \\
        \midrule
        \textbf{Precondizioni} &
        \begin{itemize}[leftmargin=*]
            \item Il configuratore è autenticato e nel menu a regime (UC V1-04).
        \end{itemize} \\
        \midrule
        \textbf{\makecell[l]{Scenario \\Principale}} &
        \begin{enumerate}[leftmargin=*]
            \item Il configuratore seleziona l'opzione per visualizzare l'elenco dei luoghi.
            \item Il sistema recupera l'elenco dei luoghi.
            \item Il sistema visualizza l'elenco, mostrando per ogni luogo almeno il nome/identificativo univoco e la collocazione.
        \end{enumerate} \\
        \midrule
        \textbf{\makecell[l]{Scenario \\Alternativo}} & \textbf{3.a}: Non ci sono luoghi inseriti. Il sistema visualizza un messaggio appropriato (es. "Nessun luogo inserito"). \\
        \midrule
        \textbf{Postcondizioni} &
        \begin{itemize}[leftmargin=*]
            \item Il configuratore ha visualizzato le informazioni richieste.
            \item Il configuratore torna al menu a regime.
        \end{itemize} \\
        \bottomrule
        \caption{UC V1-13: Elenco Luoghi} \label{uc:v1-13}
    \end{longtable}

% --- UC V1-14: Visualizzazione Elenco Tipi Visita per Luogo ---
    \newpage
    \begin{longtable}{@{} p{0.2\textwidth} p{0.75\textwidth} @{}}
        \toprule
        \rowcolor{lightgray}
        \multicolumn{2}{c}{\textbf{Use Case: Elenco Tipi Visita per Luogo}} \\
        \midrule
        \textbf{ID}        & V1-14                                                                              \\
        \midrule
        \textbf{Versione}  & 1                                                                                  \\
        \midrule
        \textbf{Attore(i)} & Configuratore (autenticato)                                                        \\
        \midrule
        \textbf{Obiettivo} & Il configuratore visualizza l'elenco dei tipi di visita associati a ciascun luogo. \\
        \midrule
        \textbf{Precondizioni} &
        \begin{itemize}[leftmargin=*]
            \item Il configuratore è autenticato e nel menu a regime (UC V1-04).
        \end{itemize} \\
        \midrule
        \textbf{\makecell[l]{Scenario \\Principale}} &
        \begin{enumerate}[leftmargin=*]
            \item Il configuratore seleziona l'opzione per visualizzare i tipi di visita per luogo.
            \item Il sistema recupera l'elenco dei luoghi e dei tipi di visita associati.
            \item Il sistema visualizza l'elenco, raggruppando per luogo e mostrando per ciascun luogo i dettagli dei tipi di visita associati (titolo, ora inizio, durata, etc.).
        \end{enumerate} \\
        \midrule
        \textbf{\makecell[l]{Scenario \\Alternativo}} & \textbf{3.a}: Non ci sono tipi di visita inseriti. Il sistema visualizza un messaggio appropriato. \\
        \midrule
        \textbf{Postcondizioni} &
        \begin{itemize}[leftmargin=*]
            \item Il configuratore ha visualizzato le informazioni richieste.
            \item Il configuratore torna al menu a regime.
        \end{itemize} \\
        \bottomrule
        \caption{UC V1-14: Elenco Tipi Visita per Luogo} \label{uc:v1-14}
    \end{longtable}

% --- UC V1-15: Visualizzazione Elenco Visite per Stato ---
    \newpage
    \begin{longtable}{@{} p{0.2\textwidth} p{0.75\textwidth} @{}}
        \toprule
        \rowcolor{lightgray}
        \multicolumn{2}{c}{\textbf{Use Case: Elenco Visite per Stato}} \\
        \midrule
        \textbf{ID}        & V1-15                                                                                                         \\
        \midrule
        \textbf{Versione}  & 1                                                                                                             \\
        \midrule
        \textbf{Attore(i)} & Configuratore (autenticato)                                                                                   \\
        \midrule
        \textbf{Obiettivo} & Il configuratore visualizza l'elenco delle visite (istanze specifiche in date specifiche) filtrate per stato. \\
        \midrule
        \textbf{Precondizioni} &
        \begin{itemize}[leftmargin=*]
            \item Il configuratore è autenticato e nel menu a regime (UC V1-04).
            \item Il sistema ha iniziato a generare piani di visite (a partire da V3, ma la visualizzazione è già prevista in V1).
        \end{itemize} \\
        \midrule
        \textbf{\makecell[l]{Scenario \\Principale}} &
        \begin{enumerate}[leftmargin=*]
            \item Il configuratore seleziona l'opzione per visualizzare le visite per stato.
            \item Il sistema chiede di specificare lo stato desiderato (proposta, completa, confermata, cancellata, effettuata) o di visualizzarle tutte.
            \item Il configuratore inserisce lo stato desiderato (o lascia vuoto per tutte).
            \item Il sistema valida lo stato inserito.
            \item Il sistema recupera le visite corrispondenti al filtro di stato.
            \item Il sistema visualizza l'elenco delle visite trovate, mostrando per ciascuna almeno: titolo del tipo visita, data, ora inizio, stato, luogo. Per le visite effettuate, mostra che appartengono all'archivio storico.
        \end{enumerate} \\
        \midrule
        \textbf{\makecell[l]{Scenario \\Alternativo}}                   & \textbf{4.a}: Lo stato inserito non è valido. Il sistema visualizza un errore e richiede nuovamente l'inserimento (torna al passo 2).       \\ \addlinespace
        & \textbf{6.a}: Non ci sono visite nello stato richiesto. Il sistema visualizza un messaggio appropriato. \\
        \midrule
        \textbf{Postcondizioni} &
        \begin{itemize}[leftmargin=*]
            \item Il configuratore ha visualizzato le informazioni richieste.
            \item Il configuratore torna al menu a regime.
        \end{itemize} \\
        \bottomrule
        \caption{UC V1-15: Elenco Visite per Stato} \label{uc:v1-15}
    \end{longtable}

    \bigskip
    \textbf{Diagrammi UML:}
% Removed [noitemsep]
    \begin{itemize}
        \item \textit{[Diagramma UML delle Classi V1 da inserire qui]}
        \item \textit{[Diagrammi Comportamentali V1 (Opzionali) da inserire qui]}
    \end{itemize}

    \newpage
%--------------------------------------------------
% Versione 2
%--------------------------------------------------

    \subsection{Versione 2: Apertura all'Accesso del Volontario}
    \textbf{Casi d'Uso Aggiuntivi/Modificati:}

% --- UC V2-01: Login (Esteso a Volontario) ---
    \begin{longtable}{@{} p{0.2\textwidth} p{0.75\textwidth} @{}}
        \toprule
        \rowcolor{lightgray}
        \multicolumn{2}{c}{\textbf{Use Case: Login Esteso (Configuratore/Volontario)}} \\
        \midrule
        \textbf{ID}        & V2-01 (Modifica V1-01)                                                                                                                    \\
        \midrule
        \textbf{Versione}  & 2                                                                                                                                         \\
        \midrule
        \textbf{Attore(i)} & Configuratore, Volontario (non autenticato)                                                                                               \\
        \midrule
        \textbf{Obiettivo} & Un utente (configuratore o volontario) accede al sistema fornendo le proprie credenziali.                                                 \\
        \midrule
        \textbf{Precondizioni} &
        \begin{itemize}[leftmargin=*]
            \item L'utente non è autenticato.
            \item Esistono credenziali (username/nickname e password) per l'utente.
        \end{itemize} \\
        \midrule
        \textbf{\makecell[l]{Scenario \\Principale}} &
        \begin{enumerate}[leftmargin=*]
            \item L'utente avvia la procedura di login.
            \item Il sistema richiede username (o nickname per volontario) e password.
            \item L'utente inserisce username/nickname e password.
            \item Il sistema verifica le credenziali e identifica il tipo di utente (Configuratore o Volontario).
            \item Le credenziali sono corrette e la password è già stata cambiata in passato.
            \item Il sistema autentica l'utente e lo identifica come Configuratore o Volontario.
        \end{enumerate} \\
        \midrule
        \textbf{\makecell[l]{Scenario \\Alternativo}}                   & \textbf{5.a}: Credenziali errate. Errore, torna al passo 2. \\ \addlinespace
        & \textbf{5.b}: Credenziali corrette ma password mai cambiata (primo accesso volontario o configuratore). Forza cambio password (UC V2-03). \\
        \midrule
        \textbf{Postcondizioni} &
        \begin{itemize}[leftmargin=*]
            \item L'utente è autenticato e il suo ruolo (Configuratore/Volontario) è noto al sistema.
            \item Se Configuratore e fase iniziale non completata, accede a UC V1-05.
            \item Altrimenti, accede al menu a regime specifico per il suo ruolo (UC V1-04 per Configuratore, UC V2-04 per Volontario).
        \end{itemize} \\
        \bottomrule
        \caption{UC V2-01: Login Esteso} \label{uc:v2-01}
    \end{longtable}

% --- UC V2-02: Cambio Password (Esteso a Volontario) ---
    \newpage
    \begin{longtable}{@{} p{0.2\textwidth} p{0.75\textwidth} @{}}
        \toprule
        \rowcolor{lightgray}
        \multicolumn{2}{c}{\textbf{Use Case: Cambio Password Esteso}} \\
        \midrule
        \textbf{ID}        & V2-02 (Modifica V1-02)                                      \\
        \midrule
        \textbf{Versione}  & 2                                                           \\
        \midrule
        \textbf{Attore(i)} & Configuratore, Volontario (autenticato)                     \\
        \midrule
        \textbf{Obiettivo} & L'utente autenticato cambia la propria password di accesso. \\
        \midrule
        \textbf{Precondizioni} &
        \begin{itemize}[leftmargin=*]
            \item L'utente (Configuratore o Volontario) è autenticato.
            \item L'utente si trova nel menu a regime del proprio ruolo.
        \end{itemize} \\
        \midrule
        \textbf{\makecell[l]{Scenario \\Principale}} &
        \begin{enumerate}[leftmargin=*]
            \item L'utente seleziona l'opzione per cambiare la password.
            \item Il sistema richiede la nuova password.
            \item L'utente inserisce la nuova password.
            \item Il sistema valida e aggiorna la password.
            \item Il sistema conferma l'avvenuto cambio.
        \end{enumerate} \\
        \midrule
        \textbf{\makecell[l]{Scenario \\Alternativo}} & \textbf{4.a}: Nuova password non valida. Errore, torna al passo 2. \\
        \midrule
        \textbf{Postcondizioni} &
        \begin{itemize}[leftmargin=*]
            \item La password dell'utente è aggiornata.
            \item L'utente torna al menu a regime del proprio ruolo.
        \end{itemize} \\
        \bottomrule
        \caption{UC V2-02: Cambio Password Esteso} \label{uc:v2-02}
    \end{longtable}

% --- UC V2-03: Cambio Password Obbligatorio (Esteso a Volontario) ---
    \newpage
    \begin{longtable}{@{} p{0.2\textwidth} p{0.75\textwidth} @{}}
        \toprule
        \rowcolor{lightgray}
        \multicolumn{2}{c}{\textbf{Use Case: Cambio Password Obbligatorio Esteso}} \\
        \midrule
        \textbf{ID}        & V2-03 (Modifica V1-03)                                                   \\
        \midrule
        \textbf{Versione}  & 2                                                                        \\
        \midrule
        \textbf{Attore(i)} & Configuratore, Volontario (autenticato)                                  \\
        \midrule
        \textbf{Obiettivo} & L'utente è obbligato a cambiare la password temporanea al primo accesso. \\
        \midrule
        \textbf{Precondizioni} &
        \begin{itemize}[leftmargin=*]
            \item L'utente si è appena autenticato per la prima volta (UC V2-01, Alt 5.b).
            \item Funzionalità limitate a cambio password e logout.
        \end{itemize} \\
        \midrule
        \textbf{\makecell[l]{Scenario \\Principale}} &
        \begin{enumerate}[leftmargin=*]
            \item Il sistema informa della necessità di cambiare password.
            \item Il sistema richiede la nuova password.
            \item L'utente inserisce la nuova password.
            \item Il sistema valida e aggiorna la password.
            \item Il sistema conferma e sblocca le funzionalità normali.
        \end{enumerate} \\
        \midrule
        \textbf{\makecell[l]{Scenario \\Alternativo}}                   & \textbf{3.a}: L'utente effettua il logout. Al login successivo, si ritrova qui.       \\ \addlinespace
        & \textbf{4.a}: Nuova password non valida. Errore, torna al passo 2. \\
        \midrule
        \textbf{Postcondizioni} &
        \begin{itemize}[leftmargin=*]
            \item Password aggiornata.
            \item L'utente accede al menu a regime del proprio ruolo (o fase iniziale se Configuratore).
        \end{itemize} \\
        \bottomrule
        \caption{UC V2-03: Cambio Password Obbligatorio Esteso} \label{uc:v2-03}
    \end{longtable}

% --- UC V2-04: Menu a Regime Volontario ---
    \newpage
    \begin{longtable}{@{} p{0.2\textwidth} p{0.75\textwidth} @{}}
        \toprule
        \rowcolor{lightgray}
        \multicolumn{2}{c}{\textbf{Use Case: Menu a Regime Volontario}} \\
        \midrule
        \textbf{ID}        & V2-04 (Nuovo)                                                                \\
        \midrule
        \textbf{Versione}  & 2                                                                            \\
        \midrule
        \textbf{Attore(i)} & Volontario (autenticato)                                                     \\
        \midrule
        \textbf{Obiettivo} & Il volontario interagisce con il menu principale specifico per il suo ruolo. \\
        \midrule
        \textbf{Precondizioni} &
        \begin{itemize}[leftmargin=*]
            \item Il volontario è autenticato (UC V2-01 o V2-03).
        \end{itemize} \\
        \midrule
        \textbf{\makecell[l]{Scenario \\Principale}} &
        \begin{enumerate}[leftmargin=*]
            \item Il sistema presenta al volontario le opzioni disponibili per la Versione 2:
            \begin{itemize} % Standard itemize
                \item Visualizzare i tipi di visita a cui è associato (UC V2-05)
                \item Esprimere/Modificare disponibilità per il mese entrante (UC V2-06)
                \item Cambiare la propria password (UC V2-02)
                \item Effettuare Logout
            \end{itemize}
            \item Il volontario seleziona un'opzione.
            \item Il sistema esegue l'azione corrispondente.
            \item Al termine dell'azione, il sistema torna a presentare il menu (passo 1).
        \end{enumerate} \\
        \midrule
        \textbf{\makecell[l]{Scenario \\Alternativo}} & \textbf{2.a}: Il volontario seleziona Logout. Sessione terminata. \\
        \midrule
        \textbf{Postcondizioni} &
        \begin{itemize}[leftmargin=*]
            \item Il volontario rimane nel menu a regime fino al logout.
        \end{itemize} \\
        \bottomrule
        \caption{UC V2-04: Menu a Regime Volontario} \label{uc:v2-04}
    \end{longtable}

% --- UC V2-05: Visualizzazione Tipi Visita Associati al Volontario ---
    \newpage
    \begin{longtable}{@{} p{0.2\textwidth} p{0.75\textwidth} @{}}
        \toprule
        \rowcolor{lightgray}
        \multicolumn{2}{c}{\textbf{Use Case: Elenco Tipi Visita per Volontario}} \\
        \midrule
        \textbf{ID}        & V2-05 (Nuovo)                                                                             \\
        \midrule
        \textbf{Versione}  & 2                                                                                         \\
        \midrule
        \textbf{Attore(i)} & Volontario (autenticato)                                                                  \\
        \midrule
        \textbf{Obiettivo} & Il volontario visualizza l'elenco dei tipi di visita per cui è abilitato a fare da guida. \\
        \midrule
        \textbf{Precondizioni} &
        \begin{itemize}[leftmargin=*]
            \item Il volontario è autenticato e nel menu a regime (UC V2-04).
        \end{itemize} \\
        \midrule
        \textbf{\makecell[l]{Scenario \\Principale}} &
        \begin{enumerate}[leftmargin=*]
            \item Il volontario seleziona l'opzione per visualizzare i propri tipi di visita.
            \item Il sistema recupera i tipi di visita associati al volontario loggato.
            \item Il sistema visualizza l'elenco, mostrando per ogni tipo visita almeno il titolo e il luogo associato.
        \end{enumerate} \\
        \midrule
        \textbf{\makecell[l]{Scenario \\Alternativo}} & \textbf{3.a}: Il volontario non è associato a nessun tipo di visita (caso anomalo, dovrebbe esserlo per esistere). Messaggio appropriato. \\
        \midrule
        \textbf{Postcondizioni} &
        \begin{itemize}[leftmargin=*]
            \item Il volontario ha visualizzato le informazioni.
            \item Il volontario torna al menu a regime.
        \end{itemize} \\
        \bottomrule
        \caption{UC V2-05: Elenco Tipi Visita per Volontario} \label{uc:v2-05}
    \end{longtable}

% --- UC V2-06: Esprimi Disponibilità Volontario ---
    \newpage
    \begin{longtable}{@{} p{0.2\textwidth} p{0.75\textwidth} @{}}
        \toprule
        \rowcolor{lightgray}
        \multicolumn{2}{c}{\textbf{Use Case: Esprimi Disponibilità Volontario}} \\
        \midrule
        \textbf{ID}        & V2-06 (Nuovo)                                                                                                                                 \\
        \midrule
        \textbf{Versione}  & 2                                                                                                                                             \\
        \midrule
        \textbf{Attore(i)} & Volontario (autenticato)                                                                                                                      \\
        \midrule
        \textbf{Obiettivo} & Il volontario comunica le date del mese successivo in cui è disponibile a fare da guida.                                                      \\
        \midrule
        \textbf{Precondizioni} &
        \begin{itemize}[leftmargin=*]
            \item Il volontario è autenticato e nel menu a regime (UC V2-04).
            \item La data corrente è compresa tra il giorno 1 e il giorno 15 (incluso) del mese 'i'.
            \item Il sistema conosce le date precluse per il mese 'i+1' (impostate dal configuratore in V1/V3) e i tipi di visita associati al volontario con i relativi giorni/periodi di programmabilità.
        \end{itemize} \\
        \midrule
        \textbf{\makecell[l]{Scenario \\Principale}} &
        \begin{enumerate}[leftmargin=*]
            \item Il volontario seleziona l'opzione per esprimere/modificare le disponibilità.
            \item Il sistema informa che le disponibilità si riferiscono all'intero mese 'i+1'.
            \item Il sistema presenta un meccanismo per inserire/selezionare le date di disponibilità per il mese 'i+1' (es. calendario, lista date).
            \item Il volontario inserisce/seleziona una o più date del mese 'i+1'.
            \item Per ogni data inserita, il sistema valida che:
            \begin{itemize} % Standard itemize
                \item La data sia nel mese 'i+1'.
                \item La data non sia preclusa a livello globale (impostata da UC V1-11).
                \item Il giorno della settimana corrisponda a un giorno in cui almeno un tipo di visita associato al volontario è programmabile.
                \item Il periodo dell'anno del tipo visita includa la data.
            \end{itemize}
            \item Le date valide vengono salvate come disponibilità del volontario per il mese 'i+1'.
            \item Il volontario conferma il completamento dell'inserimento delle disponibilità per il mese 'i+1'.
            \item Il sistema salva l'insieme completo delle disponibilità fornite (eventualmente sovrascrivendo quelle precedenti per lo stesso mese, se la modifica è permessa entro il giorno 15).
        \end{enumerate} \\
        \midrule
        \textbf{\makecell[l]{Scenario \\Alternativo}}                   & \textbf{5.a}: Una data inserita non è valida per uno dei motivi al passo 5. Il sistema segnala l'errore per quella specifica data e non la salva, permettendo al volontario di correggerla o inserirne altre. \\ \addlinespace
        & \textbf{7.a}: Il volontario non inserisce alcuna disponibilità entro il giorno 15. Si assume che non sia disponibile per l'intero mese 'i+1'. \\
        \midrule
        \textbf{Postcondizioni} &
        \begin{itemize}[leftmargin=*]
            \item Le disponibilità del volontario per il mese 'i+1' sono salvate persistentemente.
            \item Queste informazioni verranno usate nella pianificazione delle visite (V3).
            \item Il volontario torna al menu a regime.
        \end{itemize} \\
        \bottomrule
        \caption{UC V2-06: Esprimi Disponibilità Volontario} \label{uc:v2-06}
    \end{longtable}

    \bigskip
    \textbf{Aggiornamento della Documentazione:}
% Removed [noitemsep]
    \begin{itemize}
        \item Aggiornamento dei casi d’uso e dei diagrammi UML per evidenziare le modifiche apportate.
    \end{itemize}

    \bigskip
    \textbf{Diagrammi UML:}
% Removed [noitemsep]
    \begin{itemize}
        \item \textit{[Diagramma UML delle Classi V2 aggiornato da inserire qui]}
        \item \textit{[Diagramma UML dei Casi d'Uso V2 aggiornato da inserire qui]}
        \item \textit{[Diagrammi Comportamentali V2 (Opzionali) da inserire qui]}
    \end{itemize}

    \newpage
%--------------------------------------------------
% Versione 3
%--------------------------------------------------

    \subsection{Versione 3: Pianificazione Mensile e Gestione Aggiunte/Rimozioni}
    \textbf{Casi d'Uso Aggiuntivi/Modificati:}

% --- UC V3-01: Genera Piano Visite Mensile ---
    \begin{longtable}{@{} p{0.2\textwidth} p{0.75\textwidth} @{}}
        \toprule
        \rowcolor{lightgray}
        \multicolumn{2}{c}{\textbf{Use Case: Genera Piano Visite Mensile}} \\
        \midrule
        \textbf{ID}        & V3-01 (Nuovo)                                                                                                          \\
        \midrule
        \textbf{Versione}  & 3                                                                                                                      \\
        \midrule
        \textbf{Attore(i)} & Configuratore (autenticato)                                                                                            \\
        \midrule
        \textbf{Obiettivo} & Il configuratore avvia la generazione del piano delle visite proponibili per il mese successivo.                       \\
        \midrule
        \textbf{Precondizioni} &
        \begin{itemize}[leftmargin=*]
            \item Il configuratore è autenticato e nel menu a regime.
            \item È il giorno 16 del mese 'i' (o il primo giorno feriale successivo).
            \item La raccolta delle disponibilità dei volontari per il mese 'i+1' è chiusa (idealmente conclusa il giorno 15).
            \item Sono note le disponibilità dei volontari per il mese 'i+1' (da UC V2-06).
            \item Sono note le date precluse per il mese 'i+1' (impostate in un ciclo precedente tramite UC V1-11).
            \item Esistono tipi di visita e volontari associati nel sistema.
        \end{itemize} \\
        \midrule
        \textbf{\makecell[l]{Scenario \\Principale}} &
        \begin{enumerate}[leftmargin=*]
            \item Il configuratore seleziona l'opzione per chiudere la raccolta disponibilità e generare il piano visite per il mese 'i+1'.
            \item Il sistema verifica che sia il momento corretto (giorno 16+).
            \item Il sistema chiude formalmente la raccolta delle disponibilità per il mese 'i+1'.
            \item Il sistema esegue l'algoritmo di pianificazione considerando:
            \begin{itemize} % Standard itemize
                \item Tipi di visita esistenti e loro regole (periodo, giorni settimana, orario).
                \item Disponibilità dei volontari associati per ogni data del mese 'i+1'.
                \item Date precluse globalmente per il mese 'i+1'.
                \item Vincoli: max 1 istanza/tipo visita/giorno; max 1 visita/volontario/giorno.
            \end{itemize}
            \item L'algoritmo produce un insieme di visite proponibili per il mese 'i+1', ciascuna con: tipo visita, data specifica, volontario assegnato come guida. L'obiettivo è massimizzare (idealmente) il numero di visite proponibili.
            \item Il sistema salva queste visite nello stato "proposta".
            \item Il sistema conferma la generazione del piano.
            \item Il sistema abilita le operazioni di aggiunta/rimozione (UC V3-02 a V3-07).
        \end{enumerate} \\
        \midrule
        \textbf{\makecell[l]{Scenario \\Alternativo}}                   & \textbf{2.a}: Non è il momento corretto per la pianificazione. Messaggio di errore. \\ \addlinespace
        & \textbf{5.a}: Non è possibile generare alcuna visita (es. nessuna disponibilità). Il sistema informa il configuratore. \\
        \midrule
        \textbf{Postcondizioni} &
        \begin{itemize}[leftmargin=*]
            \item Le visite proponibili per il mese 'i+1' sono create e salvate nello stato "proposta".
            \item Le iscrizioni per queste visite sono aperte (gestito da V4).
            \item Le disponibilità dei volontari e le date precluse per il mese 'i+1' possono essere archiviate/eliminate dopo questa fase.
            \item Il configuratore può procedere con aggiunte/rimozioni per il mese 'i+2'.
        \end{itemize} \\
        \bottomrule
        \caption{UC V3-01: Genera Piano Visite Mensile} \label{uc:v3-01}
    \end{longtable}

% --- UC V3-02: Aggiungi Luogo (Regime) ---
    \newpage
    \begin{longtable}{@{} p{0.2\textwidth} p{0.75\textwidth} @{}}
        \toprule
        \rowcolor{lightgray}
        \multicolumn{2}{c}{\textbf{Use Case: Aggiungi Luogo (Regime)}} \\
        \midrule
        \textbf{ID}        & V3-02 (Estensione V1-08)                                                       \\
        \midrule
        \textbf{Versione}  & 3                                                                              \\
        \midrule
        \textbf{Attore(i)} & Configuratore (autenticato)                                                    \\
        \midrule
        \textbf{Obiettivo} & Il configuratore aggiunge un nuovo luogo visitabile durante la fase di regime. \\
        \midrule
        \textbf{Precondizioni} &
        \begin{itemize}[leftmargin=*]
            \item Il configuratore è autenticato e nel menu a regime.
            \item È il giorno 16 del mese 'i' (o primo feriale succ.), dopo la generazione del piano (UC V3-01) e prima della riapertura della raccolta disponibilità (UC V3-08).
        \end{itemize} \\
        \midrule
        \textbf{\makecell[l]{Scenario \\Principale}} &
        \begin{enumerate}[leftmargin=*]
            \item Il configuratore seleziona l'opzione per aggiungere un luogo.
            \item Il sistema richiede i dati del luogo (vedi UC V1-08).
            \item Il configuratore inserisce i dati.
            \item Il sistema valida i dati (vedi UC V1-08).
            \item Il sistema salva il nuovo luogo. L'effetto di questa aggiunta sarà visibile a partire dal mese 'i+2'.
            \item Il sistema richiede di associare almeno un tipo di visita e un volontario (UC V3-03, V3-04).
            \item Il sistema conferma l'aggiunta.
        \end{enumerate} \\
        \midrule
        \textbf{\makecell[l]{Scenario \\Alternativo}} & \textbf{4.a}: Dati non validi. Errore (vedi UC V1-08). \\
        \midrule
        \textbf{Postcondizioni} &
        \begin{itemize}[leftmargin=*]
            \item Il nuovo luogo è aggiunto al sistema.
            \item Il configuratore torna al menu delle operazioni del giorno 16.
        \end{itemize} \\
        \bottomrule
        \caption{UC V3-02: Aggiungi Luogo (Regime)} \label{uc:v3-02}
    \end{longtable}

% --- UC V3-03: Aggiungi Tipo Visita (Regime) ---
    \newpage
    \begin{longtable}{@{} p{0.2\textwidth} p{0.75\textwidth} @{}}
        \toprule
        \rowcolor{lightgray}
        \multicolumn{2}{c}{\textbf{Use Case: Aggiungi Tipo Visita (Regime)}} \\
        \midrule
        \textbf{ID}        & V3-03 (Estensione V1-09)                                                                          \\
        \midrule
        \textbf{Versione}  & 3                                                                                                 \\
        \midrule
        \textbf{Attore(i)} & Configuratore (autenticato)                                                                       \\
        \midrule
        \textbf{Obiettivo} & Il configuratore aggiunge un nuovo tipo di visita a un luogo esistente durante la fase di regime. \\
        \midrule
        \textbf{Precondizioni} &
        \begin{itemize}[leftmargin=*]
            \item Il configuratore è autenticato e nel menu a regime.
            \item È il giorno 16 del mese 'i' (o primo feriale succ.), dopo UC V3-01 e prima di UC V3-08.
            \item Esiste almeno un luogo nel sistema.
        \end{itemize} \\
        \midrule
        \textbf{\makecell[l]{Scenario \\Principale}} &
        \begin{enumerate}[leftmargin=*]
            \item Il configuratore seleziona l'opzione per aggiungere un tipo di visita.
            \item Il sistema chiede di selezionare il luogo.
            \item Il configuratore seleziona il luogo.
            \item Il sistema richiede i dati del tipo visita (vedi UC V1-09).
            \item Il configuratore inserisce i dati.
            \item Il sistema valida i dati (vedi UC V1-09, inclusa non sovrapposizione oraria).
            \item Il sistema salva il nuovo tipo di visita associato al luogo. L'effetto sarà visibile da mese 'i+2'.
            \item Il sistema richiede di associare almeno un volontario (nuovo o esistente) a questo tipo di visita (UC V3-04).
            \item Il sistema conferma l'aggiunta.
        \end{enumerate} \\
        \midrule
        \textbf{\makecell[l]{Scenario \\Alternativo}}                   & \textbf{3.a}: Luogo non esistente. Errore.                                            \\ \addlinespace
        & \textbf{6.a}: Dati non validi. Errore (vedi UC V1-09). \\
        \midrule
        \textbf{Postcondizioni} &
        \begin{itemize}[leftmargin=*]
            \item Il nuovo tipo di visita è aggiunto e associato al luogo.
            \item Il configuratore torna al menu delle operazioni del giorno 16.
        \end{itemize} \\
        \bottomrule
        \caption{UC V3-03: Aggiungi Tipo Visita (Regime)} \label{uc:v3-03}
    \end{longtable}

% --- UC V3-04: Aggiungi Associazione Volontario-TipoVisita (Regime) ---
    \newpage
    \begin{longtable}{@{} p{0.2\textwidth} p{0.75\textwidth} @{}}
        \toprule
        \rowcolor{lightgray}
        \multicolumn{2}{c}{\textbf{Use Case: Aggiungi Associazione Volontario (Regime)}} \\
        \midrule
        \textbf{ID}        & V3-04 (Estensione V1-10)                                                                                            \\
        \midrule
        \textbf{Versione}  & 3                                                                                                                   \\
        \midrule
        \textbf{Attore(i)} & Configuratore (autenticato)                                                                                         \\
        \midrule
        \textbf{Obiettivo} & Il configuratore associa un volontario (nuovo o esistente) a un tipo di visita esistente durante la fase di regime. \\
        \midrule
        \textbf{Precondizioni} &
        \begin{itemize}[leftmargin=*]
            \item Il configuratore è autenticato e nel menu a regime.
            \item È il giorno 16 del mese 'i' (o primo feriale succ.), dopo UC V3-01 e prima di UC V3-08.
            \item Esiste almeno un tipo di visita.
        \end{itemize} \\
        \midrule
        \textbf{\makecell[l]{Scenario \\Principale}} &
        \begin{enumerate}[leftmargin=*]
            \item Il configuratore seleziona l'opzione per associare un volontario a un tipo visita.
            \item Il sistema chiede di selezionare il tipo di visita.
            \item Il configuratore seleziona il tipo di visita.
            \item Il sistema chiede se associare un volontario nuovo o esistente.
            \item \textbf{Caso Nuovo Volontario:} (Segue UC V1-10, passi 3.a-3.d) Crea nuovo volontario.
            \item \textbf{Caso Volontario Esistente:} (Segue UC V1-10, passi 4.a-4.b) Seleziona volontario esistente.
            \item Il sistema valida che volontario (nuovo o selezionato) e tipo visita esistano e non siano già associati.
            \item Il sistema salva l'associazione. L'effetto sarà visibile da mese 'i+2'.
            \item Il sistema conferma l'operazione.
        \end{enumerate} \\
        \midrule
        \textbf{\makecell[l]{Scenario \\Alternativo}}                   & \textbf{3.a}: Tipo visita non esistente. Errore.                                        \\ \addlinespace
        & \textbf{5/6.a}: Errore nella creazione/selezione volontario (vedi UC V1-10).                                                    \\ \addlinespace
        & \textbf{7.a}: Associazione già esistente. Messaggio informativo. \\
        \midrule
        \textbf{Postcondizioni} &
        \begin{itemize}[leftmargin=*]
            \item L'associazione è creata/salvata.
            \item Il configuratore torna al menu delle operazioni del giorno 16.
        \end{itemize} \\
        \bottomrule
        \caption{UC V3-04: Aggiungi Associazione Volontario (Regime)} \label{uc:v3-04}
    \end{longtable}

% --- UC V3-05: Rimuovi Luogo ---
    \newpage
    \begin{longtable}{@{} p{0.2\textwidth} p{0.75\textwidth} @{}}
        \toprule
        \rowcolor{lightgray}
        \multicolumn{2}{c}{\textbf{Use Case: Rimuovi Luogo}} \\
        \midrule
        \textbf{ID}        & V3-05 (Nuovo)                                                          \\
        \midrule
        \textbf{Versione}  & 3                                                                      \\
        \midrule
        \textbf{Attore(i)} & Configuratore (autenticato)                                            \\
        \midrule
        \textbf{Obiettivo} & Il configuratore rimuove un luogo esistente e tutte le sue dipendenze. \\
        \midrule
        \textbf{Precondizioni} &
        \begin{itemize}[leftmargin=*]
            \item Il configuratore è autenticato e nel menu a regime.
            \item È il giorno 16 del mese 'i' (o primo feriale succ.), dopo UC V3-01 e prima di UC V3-08.
        \end{itemize} \\
        \midrule
        \textbf{\makecell[l]{Scenario \\Principale}} &
        \begin{enumerate}[leftmargin=*]
            \item Il configuratore seleziona l'opzione per rimuovere un luogo.
            \item Il sistema chiede di selezionare/identificare il luogo da rimuovere.
            \item Il configuratore seleziona il luogo.
            \item Il sistema chiede conferma dell'operazione, avvisando degli effetti collaterali.
            \item Il configuratore conferma.
            \item Il sistema attua la rimozione (con effetto dal mese 'i+2'):
            \begin{itemize} % Standard itemize
                \item Rimuove tutti i tipi di visita associati al luogo.
                \item Per ogni tipo di visita rimosso, elimina le associazioni con i volontari.
                \item \textbf{Effetto Collaterale Ricorsivo:} Se un volontario rimane senza alcun tipo di visita associato a seguito di queste rimozioni, il sistema rimuove anche quel volontario (vedi UC V3-07 per dettagli).
            \end{itemize}
            \item Il sistema rimuove il luogo dal database.
            \item Il sistema conferma l'avvenuta rimozione.
        \end{enumerate} \\
        \midrule
        \textbf{\makecell[l]{Scenario \\Alternativo}}                   & \textbf{3.a}: Luogo selezionato non esistente. Errore. \\ \addlinespace
        & \textbf{5.a}: Il configuratore annulla l'operazione. Nessuna modifica. \\
        \midrule
        \textbf{Postcondizioni} &
        \begin{itemize}[leftmargin=*]
            \item Il luogo, i suoi tipi di visita e le associazioni relative sono rimossi.
            \item Eventuali volontari rimasti senza associazioni sono rimossi.
            \item Il configuratore torna al menu delle operazioni del giorno 16.
        \end{itemize} \\
        \bottomrule
        \caption{UC V3-05: Rimuovi Luogo} \label{uc:v3-05}
    \end{longtable}

% --- UC V3-06: Rimuovi Tipo Visita ---
    \newpage
    \begin{longtable}{@{} p{0.2\textwidth} p{0.75\textwidth} @{}}
        \toprule
        \rowcolor{lightgray}
        \multicolumn{2}{c}{\textbf{Use Case: Rimuovi Tipo Visita}} \\
        \midrule
        \textbf{ID}        & V3-06 (Nuovo)                                                                  \\
        \midrule
        \textbf{Versione}  & 3                                                                              \\
        \midrule
        \textbf{Attore(i)} & Configuratore (autenticato)                                                    \\
        \midrule
        \textbf{Obiettivo} & Il configuratore rimuove un tipo di visita esistente e gestisce le dipendenze. \\
        \midrule
        \textbf{Precondizioni} &
        \begin{itemize}[leftmargin=*]
            \item Il configuratore è autenticato e nel menu a regime.
            \item È il giorno 16 del mese 'i' (o primo feriale succ.), dopo UC V3-01 e prima di UC V3-08.
        \end{itemize} \\
        \midrule
        \textbf{\makecell[l]{Scenario \\Principale}} &
        \begin{enumerate}[leftmargin=*]
            \item Il configuratore seleziona l'opzione per rimuovere un tipo di visita.
            \item Il sistema chiede di selezionare/identificare il tipo di visita da rimuovere (specificando eventualmente il luogo).
            \item Il configuratore seleziona il tipo di visita.
            \item Il sistema chiede conferma, avvisando degli effetti collaterali.
            \item Il configuratore conferma.
            \item Il sistema attua la rimozione (con effetto dal mese 'i+2'):
            \begin{itemize} % Standard itemize
                \item Rimuove l'associazione tra il tipo visita e il suo luogo.
                \item Rimuove le associazioni tra il tipo visita e tutti i volontari associati.
                \item \textbf{Effetto Collaterale Ricorsivo 1:} Se il luogo a cui era associato il tipo visita rimane senza altri tipi di visita, il sistema rimuove anche quel luogo (vedi UC V3-05).
                \item \textbf{Effetto Collaterale Ricorsivo 2:} Se un volontario rimane senza alcun tipo di visita associato a seguito di questa rimozione, il sistema rimuove anche quel volontario (vedi UC V3-07).
            \end{itemize}
            \item Il sistema rimuove il tipo di visita dal database.
            \item Il sistema conferma l'avvenuta rimozione.
        \end{enumerate} \\
        \midrule
        \textbf{\makecell[l]{Scenario \\Alternativo}}                   & \textbf{3.a}: Tipo visita selezionato non esistente. Errore.         \\ \addlinespace
        & \textbf{5.a}: Il configuratore annulla l'operazione. Nessuna modifica. \\
        \midrule
        \textbf{Postcondizioni} &
        \begin{itemize}[leftmargin=*]
            \item Il tipo di visita e le sue associazioni sono rimossi.
            \item Eventuali luoghi e/o volontari rimasti senza associazioni sono rimossi.
            \item Il configuratore torna al menu delle operazioni del giorno 16.
        \end{itemize} \\
        \bottomrule
        \caption{UC V3-06: Rimuovi Tipo Visita} \label{uc:v3-06}
    \end{longtable}

% --- UC V3-07: Rimuovi Volontario ---
    \newpage
    \begin{longtable}{@{} p{0.2\textwidth} p{0.75\textwidth} @{}}
        \toprule
        \rowcolor{lightgray}
        \multicolumn{2}{c}{\textbf{Use Case: Rimuovi Volontario}} \\
        \midrule
        \textbf{ID}        & V3-07 (Nuovo)                                                              \\
        \midrule
        \textbf{Versione}  & 3                                                                          \\
        \midrule
        \textbf{Attore(i)} & Configuratore (autenticato)                                                \\
        \midrule
        \textbf{Obiettivo} & Il configuratore rimuove un volontario esistente e gestisce le dipendenze. \\
        \midrule
        \textbf{Precondizioni} &
        \begin{itemize}[leftmargin=*]
            \item Il configuratore è autenticato e nel menu a regime.
            \item È il giorno 16 del mese 'i' (o primo feriale succ.), dopo UC V3-01 e prima di UC V3-08.
        \end{itemize} \\
        \midrule
        \textbf{\makecell[l]{Scenario \\Principale}} &
        \begin{enumerate}[leftmargin=*]
            \item Il configuratore seleziona l'opzione per rimuovere un volontario.
            \item Il sistema chiede di selezionare/identificare il volontario (nickname) da rimuovere.
            \item Il configuratore seleziona il volontario.
            \item Il sistema chiede conferma, avvisando degli effetti collaterali.
            \item Il configuratore conferma.
            \item Il sistema attua la rimozione (con effetto dal mese 'i+2'):
            \begin{itemize} % Standard itemize
                \item Rimuove tutte le associazioni tra il volontario e i tipi di visita a cui era associato.
                \item \textbf{Effetto Collaterale Ricorsivo 1:} Se un tipo di visita rimane senza alcun volontario associato a seguito di queste rimozioni, il sistema rimuove anche quel tipo di visita (vedi UC V3-06).
                \item \textbf{Effetto Collaterale Ricorsivo 2:} Se la rimozione di un tipo di visita (causata da 6.b) fa sì che un luogo rimanga senza tipi di visita, il sistema rimuove anche quel luogo (vedi UC V3-05).
            \end{itemize}
            \item Il sistema rimuove il volontario dal database.
            \item Il sistema impedisce futuri login a questo volontario.
            \item Il sistema conferma l'avvenuta rimozione.
        \end{enumerate} \\
        \midrule
        \textbf{\makecell[l]{Scenario \\Alternativo}}                   & \textbf{3.a}: Volontario selezionato non esistente. Errore.     \\ \addlinespace
        & \textbf{5.a}: Il configuratore annulla l'operazione. Nessuna modifica. \\
        \midrule
        \textbf{Postcondizioni} &
        \begin{itemize}[leftmargin=*]
            \item Il volontario e le sue associazioni sono rimossi.
            \item Eventuali tipi di visita e luoghi rimasti senza associazioni a cascata sono rimossi.
            \item Al volontario è interdetto l'accesso futuro.
            \item Il configuratore torna al menu delle operazioni del giorno 16.
        \end{itemize} \\
        \bottomrule
        \caption{UC V3-07: Rimuovi Volontario} \label{uc:v3-07}
    \end{longtable}

% --- UC V3-08: Riapri Raccolta Disponibilità ---
    \newpage
    \begin{longtable}{@{} p{0.2\textwidth} p{0.75\textwidth} @{}}
        \toprule
        \rowcolor{lightgray}
        \multicolumn{2}{c}{\textbf{Use Case: Riapri Raccolta Disponibilità}} \\
        \midrule
        \textbf{ID}        & V3-08 (Nuovo)                                                                                                             \\
        \midrule
        \textbf{Versione}  & 3                                                                                                                         \\
        \midrule
        \textbf{Attore(i)} & Configuratore (autenticato)                                                                                               \\
        \midrule
        \textbf{Obiettivo} & Il configuratore riapre la raccolta delle disponibilità dei volontari per il mese successivo a quello appena pianificato. \\
        \midrule
        \textbf{Precondizioni} &
        \begin{itemize}[leftmargin=*]
            \item Il configuratore è autenticato e nel menu a regime.
            \item È il giorno 16 del mese 'i' (o primo feriale succ.).
            \item La generazione del piano per il mese 'i+1' è stata completata (UC V3-01).
            \item Le eventuali operazioni di aggiunta/rimozione per il mese 'i+2' sono state completate (UC V3-02 a V3-07).
        \end{itemize} \\
        \midrule
        \textbf{\makecell[l]{Scenario \\Principale}} &
        \begin{enumerate}[leftmargin=*]
            \item Il configuratore seleziona l'opzione per riaprire la raccolta delle disponibilità.
            \item Il sistema abilita la funzionalità per i volontari (non rimossi) di inserire le disponibilità (UC V2-06) per il mese 'i+2'.
            \item Il sistema conferma l'apertura della raccolta.
        \end{enumerate} \\
        \midrule
        \textbf{\makecell[l]{Scenario \\Alternativo}} & Nessuno significativo. \\
        \midrule
        \textbf{Postcondizioni} &
        \begin{itemize}[leftmargin=*]
            \item I volontari possono iniziare a inserire le disponibilità per il mese 'i+2'.
            \item Il configuratore torna al menu a regime.
        \end{itemize} \\
        \bottomrule
        \caption{UC V3-08: Riapri Raccolta Disponibilità} \label{uc:v3-08}
    \end{longtable}

    \bigskip
    \textbf{Diagrammi UML:}
% Removed [noitemsep]
    \begin{itemize}
        \item \textit{[Diagramma UML delle Classi V3 aggiornato da inserire qui]}
        \item \textit{[Diagramma UML dei Casi d'Uso V3 aggiornato da inserire qui]}
        \item \textit{[Diagrammi Comportamentali V3 (Opzionali) da inserire qui]}
    \end{itemize}

    \newpage
%--------------------------------------------------
% Versione 4
%--------------------------------------------------

    \subsection{Versione 4: Accesso del Fruitore e Gestione Iscrizioni}
    \textbf{Nuove Funzionalità:}
% Removed [noitemsep]
    \begin{itemize}
        \item Registrazione e autenticazione del fruitore.
        \item Procedura di iscrizione e disdetta alle visite.
        \item Visualizzazione dello stato delle visite (proposte, confermate, cancellate) per utenti e per volontari.
    \end{itemize}

    \bigskip % Added space
    \textbf{Aggiornamento dei Casi d'Uso e UML:}
% Removed [noitemsep]
    \begin{itemize}
        \item \textit{[Descrizione testuale dei nuovi UC/modifiche per V4 da inserire qui]}
        \item \textit{[Diagramma UML delle Classi V4 aggiornato da inserire qui]}
        \item \textit{[Diagramma UML dei Casi d'Uso V4 aggiornato da inserire qui]}
        \item \textit{[Diagrammi Comportamentali V4 (Opzionali) da inserire qui]}
    \end{itemize}

    \newpage


    \section{Manuale di Installazione e Uso (Ultima Versione)}

    \subsection{Requisiti di Sistema}
% Removed [noitemsep]
    \begin{itemize}
        \item Ambiente: Java Development Kit (JDK), versione 21 o superiore.
        \item Sistema Operativo: Testato su Windows 10/11, Linux (Ubuntu/Debian), macOS. Dovrebbe funzionare su qualsiasi OS con una JVM compatibile.
    \end{itemize}

    \subsection{Istruzioni per l'Installazione ed Esecuzione}
    L'applicazione è distribuita come file JAR eseguibile. Non è richiesta un'installazione tradizionale.
% Removed [noitemsep]
    \begin{enumerate}
        \item Assicurarsi di avere Java 21 (o superiore) installato e configurato nel PATH di sistema.
        \item Aprire un terminale o prompt dei comandi nella directory contenente il file `App.jar`.
        \item Eseguire il programma con il comando:
        \begin{lstlisting}[language=bash, numbers=none, frame=single, rulecolor=\color{lightgray}]
java -jar App.jar [OPZIONE]
        \end{lstlisting}
        \item Sostituire `[OPZIONE]` con una delle opzioni disponibili (vedi sotto). Se nessuna opzione è specificata, viene usata l'opzione `D` (Default).
    \end{enumerate}

    \textbf{Opzioni di Avvio:}
% Removed [noitemsep]
    \begin{itemize}
        \item \texttt{-S}: \textbf{Setup Base.} Completa la fase di inizializzazione con dati preimpostati minimi (l'ambito territoriale, il numero massimo di iscrizioni per fruitore, 1 configuratore, 1 luogo con il relativo tipo visita e un volontario associato). Utile per iniziare subito la configurazione manuale senza passare dalla fase di setup iniziale obbligatoria.
        \item \texttt{-I}: \textbf{Inizializzazione Dati.} Esegue l'opzione \texttt{-S} e aggiunge un set di dati più ricco: 13 Luoghi, 19 Tipi Visita, 3 configuratori, 5 volontari. Utile per testare le funzionalità di visualizzazione e gestione su un dataset popolato.
        \item \texttt{-A}: \textbf{Aggiungi Disponibilità.} Esegue l'opzione \texttt{-I} e aggiunge disponibilità casuali per i volontari per il mese successivo. Utile per testare la generazione del piano visite.
        \item \texttt{-D}: \textbf{Versione Dimostrazione.} Esegue l'opzione \texttt{-A} e genera il piano visite. Simula un'istanza dell'applicazione già operativa ("a regime") con visite proposte a cui i fruitori (da creare manualmente) possono iscriversi.
        \item \texttt{-B}: \textbf{Base.} Avvio standard senza dati precaricati. L'utente (il primo configuratore) dovrà eseguire manualmente tutta la fase di inizializzazione (UC V1-05). Questa è l'opzione usata se non ne viene specificata nessuna.
    \end{itemize}

    \subsection{Guida all'Uso -- Linea di Comando}
    L'applicazione opera tramite un'interfaccia a riga di comando (CLI). Dopo l'avvio, verrà presentato un prompt. I comandi disponibili dipendono dallo stato dell'applicazione (es. fase iniziale vs regime) e dal ruolo dell'utente loggato (Configuratore, Volontario, Fruitore).

    \textbf{Comando Fondamentale:}
% Removed [noitemsep]
    \begin{itemize}
        \item \texttt{help}: Visualizza l'elenco dei comandi disponibili nel contesto attuale.
        \item \texttt{help [comando]}: Mostra la sintassi e una breve descrizione per uno specifico `[comando]`. Esempio: `help login`.
    \end{itemize}

    \textbf{Esempi Comandi Comuni (lista non esaustiva):}
    \begin{lstlisting}[language=bash, numbers=none, frame=lines, rulecolor=\color{lightgray}]
# Esempio Login
login nome_utente password_utente

# Esempio Aggiunta Utente (solo Configuratore)
# Aggiunge un nuovo configuratore
add -c nuovo_config_user nuova_config_psw
# Aggiunge un nuovo volontario
add -v nuovo_vol_nick nuova_vol_psw

# Esempio registrazione Fruitore (solo Fruitore)
login nome_fruitore password_fruitore password_fruitore

# Esempio Visualizzazione (Configuratore)
list -L     # Lista dei luoghi
list -V -p  # Lista di tutte le visite proposte

# Esempio Cambio Password (tutti gli utenti loggati)
changepsw nuova_password

# Esempio Uscita
exit

    \end{lstlisting}

    \textbf{\newline Suggerimenti Pratici:}
% Removed [noitemsep]
    \begin{itemize}
        \item Utilizzare \texttt{help} frequentemente per scoprire i comandi disponibili e la loro sintassi.
        \item In \texttt{help} frequentemente per scoprire i comandi disponibili e la loro sintassi.
        \item È utile specificare il comando in combinazione con \texttt{help} per visualizzare informazioni dettagliate su quello specifico comando (ad esempio, \texttt{help add} restituisce la pagina d'aiuto per il comando \texttt{add}).
        \item Usare il comando \texttt{exit} per terminare l'applicazione. Questo assicura che tutti i dati vengano salvati correttamente su disco prima della chiusura. La chiusura forzata del terminale potrebbe causare perdita di dati non salvati.
    \end{itemize}

    \subsection{Risorse e Supporto}
% Removed [noitemsep] - Kept from previous request
    \begin{itemize}
        \item \textbf{File di Log:} L'applicazione genera un file di log (`log.log`) nella sottodirectory `data/` rispetto alla posizione del file JAR. Questo file contiene informazioni dettagliate sull'esecuzione e può essere utile per diagnosticare problemi. \textit{Nota: La directory `data/` viene creata automaticamente se non esiste.}

        \item \textbf{Librerie Utilizzate:}
        \begin{itemize}
            \item \textbf{Jackson (per gestione JSON):}
            \begin{itemize}
                \item `Jackson.core:jackson-databind:2.18.3` (Core data-binding)
                \item `Jackson.datatype:jackson-datatype-jsr310:2.18.3` (Supporto per tipi Java 8 Date/Time)
            \end{itemize}
            \item \textbf{JUnit (per Test):}
            \begin{itemize}
                \item `JUnit 5` (versione 5.0.1 utilizzata per lo sviluppo dei test unitari)
            \end{itemize}
        \end{itemize}

        \item \textbf{Contatti per Supporto Tecnico:}
        \begin{itemize}
            \item Giorgio Felappi: g.felappi004@studenti.unibs.it
            \item Daniel Barbetti: d.barbetti@studenti.unibs.it
        \end{itemize}
    \end{itemize}

    \newpage


    \section{Dimostrazione del Funzionamento dell'Applicazione}

    \subsection{Requisiti per il Salvataggio dei Dati per la Dimostrazione}
    Per facilitare la dimostrazione, l'applicazione deve essere consegnata con uno stato persistente salvato che includa almeno:
% Removed [noitemsep]
    \begin{itemize}
        \item \textbf{Credenziali Utente:}
        % Removed [noitemsep]
        \begin{itemize}
            \item Minimo 2 utenti Configuratore.
            \item Minimo 2 utenti Volontario.
            \item Minimo 3 utenti Fruitore.
        \end{itemize}
        \item \textbf{Configurazione Visite:}
        % Removed [noitemsep]
        \begin{itemize}
            \item Minimo 2 Luoghi distinti.
            \item Minimo 3 Tipologie di Visita totali, distribuite tra i luoghi. Assicurarsi che siano correttamente associate ai luoghi e ai volontari.
        \end{itemize}
        \item \textbf{Stato Visite Pianificate:}
        % Removed [noitemsep]
        \begin{itemize}
            \item Minimo 5 visite nello stato "proposta".
            \item Minimo 2 visite nello stato "confermata" (con iscrizioni sufficienti).
            \item Minimo 1 visita nello stato "cancellata" (es. per disdette).
            \item (Opzionale) Qualche visita nello stato "effettuata" (archivio storico).
        \end{itemize}
    \end{itemize}
    \textit{Nota: Assicurarsi che l'applicazione, avviata senza opzioni speciali (o con un'opzione specifica per caricare lo stato demo), carichi automaticamente questi dati.}

    \subsection{Preparazione della Demo}
% Removed [noitemsep]
    \begin{itemize}
        \item Avviare l'applicazione in modo che carichi automaticamente i dati persistenti preparati come da requisiti.
        \item Verificare tramite comandi di visualizzazione (`list`, `show`, ecc.) che i dati siano stati caricati correttamente (utenti, luoghi, tipi visita, visite pianificate con i loro stati).
    \end{itemize}

    \subsection{Sequenza della Dimostrazione}
    La dimostrazione seguirà indicativamente questa sequenza:
% Removed [noitemsep]
    \begin{enumerate}
        \item \textbf{Avvio e Stato Iniziale:}
        % Removed [noitemsep]
        \begin{itemize}
            \item Avvio dell'applicazione con caricamento dei dati demo.
            \item Breve panoramica dello stato caricato usando comandi di visualizzazione (elenco luoghi, tipi visita, visite per stato).
        \end{itemize}
        \item \textbf{Login e Ruoli:}
        % Removed [noitemsep]
        \begin{itemize}
            \item Login come Configuratore: mostrare i comandi disponibili (aggiunta/rimozione, visualizzazione completa, generazione piano - se applicabile alla demo). Eseguire un comando significativo (es. `list visite -s all`). Logout.
            \item Login come Volontario: mostrare i comandi disponibili (visualizzazione tipi visita associati, disponibilità - se demo copre questo ciclo, visualizzazione proprie visite assegnate). Logout.
            \item Login come Fruitore: mostrare i comandi disponibili (visualizzazione visite proponibili, iscrizione, visualizzazione proprie iscrizioni, disdetta).
        \end{itemize}
        \item \textbf{Flusso Principale (Iscrizione):}
        % Removed [noitemsep]
        \begin{itemize}
            \item (Come Fruitore) Visualizzare le visite `proposta` disponibili (`list visite -s proposta` o comando equivalente).
            \item (Come Fruitore) Iscriversi a una visita (`register` o comando equivalente), specificando il numero di partecipanti. Mostrare la conferma.
            \item (Come Fruitore) Visualizzare le proprie iscrizioni (`myvisits` o comando equivalente).
            \item (Opzionale) Eseguire iscrizioni multiple fino a raggiungere il numero minimo per far passare una visita a `confermata`. Mostrare il cambio di stato (login come Config/Fruitore per vedere).
            \item (Opzionale) Eseguire una disdetta (`unregister` o comando equivalente). Mostrare il cambio di stato se la visita torna `proposta` o viene `cancellata`.
        \end{itemize}
        \item \textbf{Gestione e Pianificazione (se tempo permette):}
        % Removed [noitemsep]
        \begin{itemize}
            \item (Come Configuratore) Mostrare l'aggiunta di un elemento (es. luogo o tipo visita) o la rimozione, evidenziando gli effetti (anche a cascata se implementati).
            \item (Come Configuratore) Se rilevante per la demo, mostrare il comando per generare il piano visite (`makeplan` o equivalente) o per gestire le date precluse.
        \end{itemize}
        \item \textbf{Chiusura:}
        % Removed [noitemsep]
        \begin{itemize}
            \item Uscire correttamente dall'applicazione con `exit` per mostrare il salvataggio persistente.
        \end{itemize}
    \end{enumerate}

    \subsection{Considerazioni Finali della Demo}
% Removed [noitemsep]
    \begin{itemize}
        \item Saranno brevemente menzionate eventuali limitazioni note dell'implementazione attuale o funzionalità non completamente coperte.
        \item Si preciserà che aspetti più dettagliati dell'implementazione, delle scelte progettuali e delle funzionalità complete saranno discussi durante la prova orale.
    \end{itemize}

    \newpage


    \section{Allegati e Note}

    \subsection{Diagrammi UML delle classi}
    \textit{Rappresentazione }

    \subsubsection{Versione 1}

    \subsubsection{Versione 2}

    \subsubsection{Versione 3}

    \subsubsection{Versione 4}

    \subsection{Commenti e Possibili Estensioni}
% Removed [noitemsep]
    \begin{itemize}
        \item \textbf{Estensioni Future:} Breve accenno alle possibili estensioni future menzionate nella specifica del progetto (es. notifiche email, interfaccia grafica, pagamenti online per biglietti, sistema di rating per guide/visite, ecc.) e alla fattibilità di implementarle sulla base dell'architettura attuale.
        \item \textbf{Motivazioni Progettuali:} Verranno discusse oralmente le principali scelte progettuali (es. formato di persistenza dei dati, gestione della concorrenza - se applicabile, struttura delle classi, pattern utilizzati).
    \end{itemize}

    \subsection{Riscontri e Scelte Progettuali}
    \textit{[Questa sezione può contenere note specifiche su decisioni prese durante lo sviluppo]}
% Removed [noitemsep]
    \begin{itemize}
        \item \textbf{Persistenza Dati:} Scelta del meccanismo di salvataggio (es. file di testo serializzati, JSON, CSV, database embedded come SQLite) e motivazione (semplicità, requisiti, performance).
        \item \textbf{Gestione Stati Visita:} Logica implementata per la transizione tra gli stati delle visite (proposta, confermata, cancellata, ecc.) in base alle iscrizioni/disdette e al raggiungimento del numero minimo/massimo.
        \item \textbf{Algoritmo Pianificazione:} Descrizione (anche ad alto livello) dell'algoritmo usato in UC V3-01 per generare il piano mensile, considerando vincoli e disponibilità.
        \item \textbf{Gestione Ricorsiva Rimozioni:} Dettagli sull'implementazione delle rimozioni a cascata (UC V3-05, V3-06, V3-07) per mantenere la coerenza dei dati.
    \end{itemize}

\end{document}